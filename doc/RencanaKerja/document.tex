\documentclass[a4paper,twoside]{article}
\usepackage[T1]{fontenc}
\usepackage[bahasa]{babel}
\usepackage{graphicx}
\usepackage{graphics}
\usepackage{float}
\usepackage[cm]{fullpage}
\pagestyle{myheadings}
\usepackage{etoolbox}
\usepackage{setspace} 
\usepackage{lipsum} 
\setlength{\headsep}{30pt}
\usepackage[inner=2cm,outer=2.5cm,top=2.5cm,bottom=2cm]{geometry} %margin
% \pagestyle{empty}

\makeatletter
\renewcommand{\@maketitle} {\begin{center} {\LARGE \textbf{ \textsc{\@title}} \par} \bigskip {\large \textbf{\textsc{\@author}} }\end{center} }
\renewcommand{\thispagestyle}[1]{}
\markright{\textbf{\textsc{AIF184001.03/AIF184002.05 \textemdash Rencana Kerja Skripsi \textemdash Sem. Genap 2019/2020}}}

\newcommand{\HRule}{\rule{\linewidth}{0.4mm}}
\renewcommand{\baselinestretch}{1}
\setlength{\parindent}{0 pt}
\setlength{\parskip}{6 pt}

\onehalfspacing
 
\begin{document}

\title{\@judultopik}
\author{\nama \textendash \@npm} 

%tulis nama dan NPM anda di sini:
\newcommand{\nama}{Andrianto Chandra}
\newcommand{\@npm}{2016730017}
\newcommand{\@judultopik}{Pengembangan aplikasi pemantauan getaran gedung menggunakan WSN}% Judul/topik anda
\newcommand{\jumpemb}{1} % Jumlah pembimbing, 1 atau 2
\newcommand{\tanggal}{06/02/2020}

% Dokumen hasil template ini harus dicetak bolak-balik !!!!

\maketitle

\pagenumbering{arabic}

\section{Deskripsi}
Kesehatan sebuah bangunan/gedung merupakan salah satu persyaratan teknis yang harus ada saat mengurus IMB. IMB atau yang biasa dikenal dengan Izin Mendirikan Bangunan adalah sebuah perizinan yang diberikan oleh Kepala Daerah kepada pemilik bangunan untuk membangun baru, mengubah, memperluas, mengurangi, dan/atau merawat bangunan sesuai dengan persyaratan administratif dan persyaratan teknis yang berlaku. IBM ini sangat penting khususnya untuk bangunan atau gedung yang bertingkat seperti gedung-gedung yang ada di Universitas Katolik Parahyangan. Salah satu gedung yang ada di Universitas Katolik Parahyangan adalah gedung 10 yang termasuk juga area \textit{rooftop}.


Salah satu parameter dalam pemantauan kesehatan sebuah gedung atau bangunan adalah getaran. Getaran yang terjadi pada sebuah bangunan dapat dipengeruhi oleh dua faktor. Faktor yang pertama adalah faktor dari bumi dan faktor kedua adalah faktor dari dalam gedung itu sendiri. Getaran merupakan salah satu faktor penyebab gempa bumi dimana terjadi pada kerak bumi sebagai gejala aktivitas tektonis maupun vulkanis. Pada umumnya getaran ini diakibatkan oleh adanya pergeseran lempeng pada permukaan bumi sehingga dapat terjadi gelombang gempa bumi. Getaran yang berasal dari gedung itu sendiri dapat dicontohkan dengan adanya mesin bertenaga besar yang terdapat dalam gedung tersebut seperti lift. Mesin dengan tenaga yang besar ini perlahan-lahan dapat menyebabkan sebuah gedung akan merasakan sebuah getaran yang lama kelamaan membuat gedung ini menjadi tidak stabil dan membuat kesehatan gedung menjadi memburuk. 

Seiring berkembangnya teknologi, getaran pada suatu bangunan dapat dilihat untuk memberikan kemudahan dalam melakukan pengukuran data agar menjadi lebih efektif. Proses ini digunakan sebagai pengamatan, perekaman, dan pengevaluasian dalam hal ini adalah getaran pada sebuah bangunan atau gedung untuk menilai kesehatan secara berkelanjutan. Getaran pada sebuah gedung dapat ditangkap dengan menggunakan sensor \textit{Accelerometer} yang disebar di sisi gedung yang akan membentuk sebuah jaringan sensor nirkabel.

Pada skripsi ini, akan dibuat sebuah perangkat lunak yang dapat menampilkan hasil pantauan getaran sebuah gedung yang ditampilkan dalam bentuk \textit{graph} dengan menggunakan \textit{Wireless Sensor Network} (WSN). Dengan menggunakan perangkat lunak tersebut, dapat diketahui seberapa besar getaran yang terjadi pada sebuah gedung sehingga dapat mengetahui besar kecilnya getaran yang dihasilkan tersebut.

\section{Rumusan Masalah}
Masalah-masalah yang ingin diselesaikan dalam skripsi ini adalah sebagai berikut:
\begin{itemize}
	\item Bagaimana sensor \textit{Accelerator} bekerja?
	\item Bagaimana \textit{Wireless Sensor Network} bekerja?
	\item Bagaimana komunikasi \textit{wireless} 802.15.14 (\textit{Zigbee Standard}) bekerja? 
	\item Bagaimana membangun aplikasi pemantauan getaran gedung dengan menggunakan jaringan \textit{wireless} sensor?
\end{itemize}

\section{Tujuan}
Tujuan-tujuan dari pembuatan perangkat lunak adalah sebagai berikut:
\begin{itemize}
	\item Mempelajari cara kerja sensor \textit{Accelerator}.
	\item Mempelajari cara kerja \textit{Wireless Sensor Network}.
	\item Mempelajari cara kerja \textit{wireless} 802.15.14 (\textit{Zigbee Standard}).
	\item Membangun aplikasi pemantauan getaran gedung menggunakan \textit{Wireless Sensor Network } (WSN).
\end{itemize}

\section{Deskripsi Perangkat Lunak}
Perangkat lunak akhir yang akan dibuat memiliki fitur minimal sebagai berikut:
\begin{itemize}
	\item Pengguna dapat melihat grafik getaran gedung secara kontinu yang dihasilkan oleh sensor \textit{accelerometer}.
	\item Pengguna dapat mengubah interval tangkapan getaran yang ditangkap oleh sensor.
	\item Pengguna dapat memantau getaran gedung menggunakan \textit{Wireless Sensor Network}.
	\item Pengguna dapat mengetahui apakah suatu bangunan atau gedung masih dalam kondisi sehat atau tidak.
	\item Aplikasi akan menampilkan besaran getaran sebuah gedung dengan satuan frekuensi (Hz).
\end{itemize}

\section{Detail Pengerjaan Skripsi}
Bagian-bagian pekerjaan skripsi ini adalah sebagai berikut :
	\begin{enumerate}
		\item Mempelajari bahasa pemograman Java pada sensor.
		\item Mempelajari cara kerja sensor \textit{Accelerator}.
		\item Mempelajari cara kerja \textit{Wireless Sensor Network}.
		\item Mempelajari sistem \textit{Structural Health Monitoring} (SHM).
		\item Menganalisis perangkat lunak sistem pemantauan getaran gedung yang sudah ada
		\item Mempelajari komunikasi \textit{wireless} 802.15.14 (\textit{Zigbee Standard}).
		\item Merancang infrastruktur jaringan \textit{Wireless Sensor Network} (WSN)
		\item Membangun infrastruktur jaringan \textit{Wireless Sensor Network} (WSN)
		\item Mengimplementasi kode program pada sensor Accelerator
		\item Melakukan pengujian sensor \textit{Accelerator} yang telah diimplementasikan ke gedung yang ada di Universitas Katolik Parahyangan.
		\item Menulis dokumen skripsi.
	\end{enumerate}

\section{Rencana Kerja}
Rincian capaian yang direncanakan di Skripsi 1 adalah sebagai berikut:
\begin{enumerate}
\item Mempelajari tentang \textit{Wireless Sensor Network}.
\item Mempelajari cara kerja sensor \textit{Accelerometer}.
\item Mempelajari komunikasi \textit{wireless} 802.15.14 (\textit{Zigbee Standard}).
\item Mempelajari sistem \textit{Structural Health Monitoring} (SHM).
\item Menganalisis perangkat lunak sistem pemantauan getaran gedung yang sudah ada
\item Mempelajari pemograman Java di sensor maupun \textit{desktop}.
\item Menulis dokumen skripsi bab 1, 2, dan 3.
\end{enumerate}

Sedangkan yang akan diselesaikan di Skripsi 2 adalah sebagai berikut:
\begin{enumerate}
\item Merancang infrastruktur jaringan \textit{Wireless Sensor Network} (WSN)
\item Membangun infrastruktur jaringan \textit{Wireless Sensor Network} (WSN)
\item Mengimplementasikan kode program pada sensor \textit{Accelerometer} yang digunakan untuk menngukur getaran di gedung.
\item Melakukan pengujian sensor \textit{Accelerator} yang telah diimplementasikan ke gedung yang ada di Universitas Katolik Parahyangan. 
\item Menulis dokumen skripsi hingga selesai.
\end{enumerate}

\vspace{1cm}
\centering Bandung, \tanggal\\
\vspace{2cm} \nama \\ 
\vspace{1cm}

Menyetujui, \\
\ifdefstring{\jumpemb}{2}{
\vspace{1.5cm}
\begin{centering} Menyetujui,\\ \end{centering} \vspace{0.75cm}
\begin{minipage}[b]{0.45\linewidth}
% \centering Bandung, \makebox[0.5cm]{\hrulefill}/\makebox[0.5cm]{\hrulefill}/2013 \\
\vspace{2cm} Nama: \makebox[3cm]{\hrulefill}\\ Pembimbing Utama
\end{minipage} \hspace{0.5cm}
\begin{minipage}[b]{0.45\linewidth}
% \centering Bandung, \makebox[0.5cm]{\hrulefill}/\makebox[0.5cm]{\hrulefill}/2013\\
\vspace{2cm} Nama: \makebox[3cm]{\hrulefill}\\ Pembimbing Pendamping
\end{minipage}
\vspace{0.5cm}
}{
% \centering Bandung, \makebox[0.5cm]{\hrulefill}/\makebox[0.5cm]{\hrulefill}/2013\\
\vspace{2cm} Nama: \makebox[3cm]{\hrulefill}\\ Pembimbing Tunggal
}
\end{document}

